\documentclass{article}
\usepackage{graphicx} 
\usepackage{amsmath}
\usepackage{amssymb}
\usepackage[hidelinks]{hyperref}
\usepackage{subcaption}
\usepackage{tikz}
\usepackage{braket}
\usepackage{minted}
\usepackage{pseudo}
\usepackage{mathtools}
\usepackage{tabto}
\usepackage{float}
\usepackage{titlesec}
\usetikzlibrary{shapes.geometric}
\usepackage{lastpage}
\usepackage{tocloft}
\title{Peg Solitaire}
\author{
\centering
\begin{tabular}{c}
    Asmus Tørsleff\\
    qjv778@alumni.ku.dk (asmus.torsleff@gmail.com)\\
\end{tabular}}
\date{\today}
\setcounter{tocdepth}{3}
\renewcommand{\thesection}{}
\renewcommand{\thesubsection}{\arabic{subsection}}
\renewcommand{\thesubsubsection}{\alph{subsubsection})}

\begin{document}
\maketitle
%\tableofcontents
\section{introduction}
Let $b$ be a board of size $n$, this can be represented as a $n$ bit binary number. 
A 1 in the $i$'th position in $b$ represents a peg in the $i$'th board position, a 0 represents a cavity.
$$b=b_0b_1...b_{n-2}b_{n-1},\ b_i\in \{0,1\}$$
\section*{definitions}
Let $p(b)=x$ where $x$ is the the number of ones in $b$, aka. the pop count. 

\vspace{\baselineskip}
\noindent
Let $a(b, i)=b'$ be the funtion negating the i'th bit and its neighbours. $a(b,i)$ is only defined when $b_{i} = 1$ and $b_{i-1} \neq b_{i+1}$.  
Applying $a$ equates to a move and removes a peg from the board $p(b')=p(b)-1$.

\vspace{\baselineskip}
\noindent
We can apply $a$ repeatedly to $b$ until it is no longer defined for any $i$. 
The board, $s$, we get at the end we will call a solution.
If we do this in every posible way we get a set $S(b)$ of all solutions for $b$.
The set of optimal solutions is $O(b)=\{s\in S(b)\ |\ \forall_{s'\in S(b)}\ p(s)\le p(s')\}$

\vspace{\baselineskip}
\noindent
Let $f(b,m)=b'$ be the function applying $a(b,i_j)$ $m$ times, with each application taking $b$ closer to an optimal solution. 
If $a$ can not be applied $m$ times before becoming undefined then $f(b,m) = b',\ b'\in O(b)$. 

\vspace{\baselineskip}
\noindent
Let $\cdot$ be the board concatenation operator. $b=x\cdot y=x_0x_1..x_{n-1}y_0y_1..y_{n-1} $. 

\vspace{\baselineskip}
\noindent
Let $b^x=b^{x-1}\cdot b$, where $b^0=id$ is the board such that $b =b\cdot id=id\cdot b$. 

\vspace{\baselineskip}
\noindent
Let $|b|$ be the board length operator. $|b|=n$. 

\vspace{\baselineskip}
\noindent
Let $o(b) = f(b,|b|)\in O(b)$

\vspace{\baselineskip}
\noindent
Let $x_i$ be the position of the $i$'th 1 in $b$. 
Let $w(b) = x_{p(b)}-x_1$

\vspace{\baselineskip}
\noindent
Let $r(b)$ be the function that reverses the board. $r(x\cdot y)=r(y) \cdot r(x),\ r(1) = 1,\ r(0) = 0$


\section{Conjectures}
\begin{enumerate}
    \item $|b|\ge p(b)$
    \item $p(b)\ge p\circ f(b,m)$
    \item $p\circ f(b,m)\ge p(b)-m$
    \item $p(b) \le 1 \Rightarrow f(b,m) = b$
    \item $p(b) = |b| \Rightarrow f(b,m) = b$
    \item $p(b)\ge 0$
    \item $w(b)\le |b|$ 
    \item $p(b) = p\circ r(b)$
    \item $w(b) = w\circ r(b)$
    \item $p\circ o(b) = p\circ o\circ r(b)$
    \item $b = b'\cdot 0^3\cdot b'',\quad p\circ o(b) = p\circ o(b') + p\circ o(b'')$
    \item $b = b'\cdot b'',\quad p\circ o(b) \le p\circ o(b') + p\circ o(b'')$
    \item $w\circ f(b,x)\le w(b)+2$ 
    \item $b=0\cdot 1^n\cdot 0,\ p\circ o(b) = \frac{n+1}{2}$ 
\end{enumerate}

\section{Proofs}
\subsection{$|b|\ge p(b)$}
There can never be more ones in $b$ than there are bits. 

\subsection{$p(b)\ge p\circ f(b,m)$}
Each application of $a$ removes a one. Thus $f$ can never add ones. 

\subsection{$p\circ f(b,m)\ge p(b)-m$}
Each application of $a$ removes a one. $f(b,m)$ applies $a\ m$ or less times. Thus $f(b,m)$ can never remove more than $m$ ones. 

\subsection{$p(b) \le 1 \Rightarrow f(b,m) = b$}
If $p(b) \le 0$ then there are not enough ones to apply $a$. 
Therefore $f$ will apply $a$ zero times, thus making no changes.

\subsection{$p(b) = |b| \Rightarrow f(b,m) = b$}
If $p(b) = |b| $ then there are not enough zeroes to apply $a$. 
Therefore $f$ will apply $a$ zero times, thus making no changes.

\subsection{$p(b)\ge 0$}
We can never have a negative number of ones on a board.

\subsection{$w(b)\le |b|$}
Since $w(b) = x_{p(b)}-x_1$ where $0\le x_1\le x_{p(b)} \le |b|$ then $w$ is maximized when $x_1=0$.
Then we have $w(b) =x_{p(b)}\le |b|$. 

\subsection{$p(b) = p\circ r(b)$}\label{rev_count}
Changing the order of the bits does not change the number of ones.

\subsection{$w(b) = w\circ r(b)$}
$w(b) = x_{p(b)}-x_1= (|b|-x_{1})-(|b|-x_{p(b)})=w\circ r(b)$

\subsection{$p\circ o(b) = p\circ o\circ r(b)$}
If we can apply $a(b,i)$ then we can also apply $a(r(b),|b|-i-1)$. 
So $f(b,m) = r\circ f(r(b),m)$ and then $o(b)=r\circ o\circ r(b)$ and by \ref{rev_count} we have $p\circ o(b) = p\circ r\circ o\circ r(b)=p\circ o\circ r(b)$.
\end{document}

