\documentclass{article}
\usepackage{graphicx} 
\usepackage{amsmath}
\usepackage{amssymb}
\usepackage[hidelinks]{hyperref}
\usepackage{subcaption}
\usepackage{tikz}
\usepackage{braket}
\usepackage{minted}
\usepackage{pseudo}
\usepackage{mathtools}
\usepackage{tabto}
\usepackage{float}
\usepackage{titlesec}
\usetikzlibrary{shapes.geometric}
\usepackage{lastpage}
\usepackage{tocloft}
\title{Peg Solitaire}
\author{
\centering
\begin{tabular}{c}
    Asmus Tørsleff\\
    qjv778@alumni.ku.dk (asmus.torsleff@gmail.com)\\
\end{tabular}}
\date{\today}
\setcounter{tocdepth}{3}
\renewcommand{\thesection}{}
\renewcommand{\thesubsection}{\arabic{subsection}}
\renewcommand{\thesubsubsection}{\alph{subsubsection})}

\begin{document}
\maketitle
%\tableofcontents
\section{introduction}
Let $b$ be a board of size $n$, this can be represented as a $n$ bit binary number. 
A 1 in the $i$'th position in $b$ represents a peg in the $i$'th board position, a 0 represents a cavity.
$$b=b_0b_1...b_{n-2}b_{n-1},\ b_i\in \{0,1\}$$

\section{definitions}
Let $p(b)=x$ where $x$ is the the number of ones in $b$, aka. the pop count. 

\vspace{\baselineskip}
\noindent
Let $a(b, i)=b'$ be the funtion negating the i'th bit and its neighbours. $a(b,i)$ is only defined when $b_{i} = 1$ and $b_{i-1} \neq b_{i+1}$.  
Applying $a$ equates to a move and removes a peg from the board $p(b')=p(b)-1$.

\vspace{\baselineskip}
\noindent
We can apply $a$ repeatedly to $b$ until it is no longer defined for any $i$. 
The board, $s$, we get at the end we will call a solution.
If we do this in every posible way we get a set $S(b)$ of all solutions for $b$.
The set of optimal solutions is $O(b)=\{s\in S(b)\ |\ \forall_{s'\in S(b)}\ p(s)\le p(s')\}$

\vspace{\baselineskip}
\noindent
Let $f(b,m)=b'$ be the function applying $a(b,i_j)$ $m$ times, with each application taking $b$ closer to an optimal solution. 
If $a$ can not be applied $m$ times before becoming undefined then $f(b,m) = b',\ b'\in O(b)$. 

\vspace{\baselineskip}
\noindent
Let $\cdot$ be the board concatenation operator. $b=x\cdot y=x_0x_1..x_{n-1}y_0y_1..y_{n-1} $. 

\vspace{\baselineskip}
\noindent
Let $b^x=b^{x-1}\cdot b$, where $b^0=id$ is the board such that $b =b\cdot id=id\cdot b$. 

\vspace{\baselineskip}
\noindent
Let $|b|$ be the board length operator. $|b|=n$. 

\vspace{\baselineskip}
\noindent
Let $o(b) = f(b,|b|)\in O(b)$

\vspace{\baselineskip}
\noindent
Let $z_{i}(b) = x_i$ be the position of the $i$'th 1 in $b$. $0<i\le p(b)$. 

\vspace{\baselineskip}
\noindent
Let $w(b) = z_{p(b)}(b)-z_1(b)$

\vspace{\baselineskip}
\noindent
Let $r(b)$ be the function that reverses the board. $r(x\cdot y)=r(y) \cdot r(x),\ r(1) = 1,\ r(0) = 0$


\section{Conjectures}
\begin{enumerate}
    \item $|b|\ge p(b)$
    \item $p(b)\ge p\circ f(b,m)$
    \item $p\circ f(b,m)\ge p(b)-m$
    \item $p(b) \le 1 \Rightarrow f(b,m) = b$
    \item $p(b) = |b| \Rightarrow f(b,m) = b$
    \item $p(b)\ge 0$
    \item $0 \le z_i(b)\le |b|$ 
    \item $p(b) = p\circ r(b)$
    \item $w(b) = w\circ r(b)$
    \item $p\circ o(b) = p\circ o\circ r(b)$
    \item $b = b'\cdot b'',\quad p\circ f(b,m_1+m_2) \le p\circ f(b',m_1) + p\circ f(b'',m_2)$
    \item $z_1(b)-1\le z_1(f(b,x))$
    \item $w\circ f(b,x)\le w(b)+2$ 
    \item $b = b'\cdot 0^3\cdot b'',\quad p\circ f(b,m_1+m_2) = p\circ f(b',m_1) + p\circ f(b'',m_2)$
    \item $b=0\cdot 1^n\cdot 0,\ p\circ o(b) = \lceil\frac{n}{2}\rceil$ 
\end{enumerate}

\section{Proofs}
\subsection{$|b|\ge p(b)$}
There can never be more ones in $b$ than there are bits. 

\subsection{$p(b)\ge p\circ f(b,m)$}
Each application of $a$ removes a one. Thus $f$ can never add ones. 

\subsection{$p\circ f(b,m)\ge p(b)-m$}
Each application of $a$ removes a one. $f(b,m)$ applies $a\ m$ or less times. Thus $f(b,m)$ can never remove more than $m$ ones. 

\subsection{$p(b) \le 1 \Rightarrow f(b,m) = b$}
If $p(b) \le 0$ then there are not enough ones to apply $a$. 
Therefore $f$ will apply $a$ zero times, thus making no changes.

\subsection{$p(b) = |b| \Rightarrow f(b,m) = b$}
If $p(b) = |b| $ then there are not enough zeroes to apply $a$. 
Therefore $f$ will apply $a$ zero times, thus making no changes.

\subsection{$p(b)\ge 0$}
We can never have a negative number of ones on a board.

\subsection{$0 \le z_i(b)\le |b|$}
Since we only have bits 0 to $|b|$ the $i$'th 1 can never be outside this range.

\subsection{$p(b) = p\circ r(b)$}\label{rev_count}
Changing the order of the bits does not change the number of ones.

\subsection{$w(b) = w\circ r(b)$}
$w(b) = z_{p(b)}(b)-z_1(b)= (|b|-z_{1}(b))-(|b|-z_{p(b)}(b))=w\circ r(b)$

\subsection{$p\circ o(b) = p\circ o\circ r(b)$}
If we can apply $a(b,i)$ then we can also apply $a(r(b),|b|-i-1)$. 
So $f(b,m) = r\circ f(r(b),m)$ and then $o(b)=r\circ o\circ r(b)$ and by \ref{rev_count} we have $p\circ o(b) = p\circ r\circ o\circ r(b)=p\circ o\circ r(b)$.

\subsection{$b = b'\cdot b'',\quad p\circ f(b,m_1+m_2) \le p\circ f(b',m_1) + p\circ f(b'',m_2)$ }
Any valid move order in $b'$ and $b''$ is naturally also valid in $b$. 
Thus we can accieve equality by executing the same moves. 
Sometimes though, we can do better: 
\begin{equation*}
\begin{split}
    p\circ f (01,1) = p\circ f (10,1) &= 1 \\
    p\circ f (0110,2) = p(0001) &= 1 \\
    p\circ f (01,1) + p\circ f (10,1) = 2 &\ge 1 = p\circ f (0110,2) 
\end{split}
\end{equation*}

\subsection{$z_1(b)-1\le z_1(f(b,x))$}
We want to show that a board $b = 0^{n_1}1?^{n_2}$ can never evolve into $0^{n_1-2}1?^{n_2+2}$
For $n_1<2$ this is obvious. 
For $n_1>2$ it is the same as for $n_1=2$. 
Thus we want to prove that a board $b = 001?^{n}$ can never become $1?^{n+2}$. 
\\
We start by showing that a board $b = 00?^{n}$ can never become $1?^{n+1}$. 
We do this by induction. \\
Base case $n=0$:\\
$b = 00$. there are no moves so it can not become $1?$. \\
Base case $n=1$: \\
$b \in {000, 001}$ in either case there are no moves and so it can not become $1??$. \\
Base case $n=2$: \\
$b \in {0000, 0001, 0010, 0011}$. the first 3 do not have any available moves and so can not become $1???$ the fourth becomes $0100$ after its only available move. \\
Case $n\ge 3$: \\
We can never get a 1 in the first position unless we at some point have one in the second position.
If we somehow get a 1 in the second position we know that it has to be the result of a move that cleared the third and fourth. 
So we have $b' = 0100?^{n-2}$. By the inductive hypothesis it is imposible to get a one in the third position which is requred to make a move resulting in a 1 in the first position. 

\vspace{\baselineskip}
\noindent
Now we show that a board $b = 001?^{n}$ can never become $1?^{n+2}$.  
When $n=0$:\\
$b = 001$. there are no moves so it can not become $1??$. \\
When $n=1$: \\
$b \in {0010, 0011}$ in the first case there are no moves and so it can not become $1??$. The second case becomes $0100$ after the only valid move.\\
when $n=2$: \\
$b \in {00100, 00101, 00110, 00111}$. the first 2 do not have any available moves and so can not become $1???$. The third becomes either $01000$ or $00001$ after its only available moves. The fourth becomes $01001$ after its only available move. \\
Case $n\ge 3$: \\
We can never get a 1 in the first position unless we at some point have one in the second position.
If we somehow get a 1 in the second position we know that it has to be the result of a move that cleared the third and fourth. 
So we have $b' = 0100?^{n-2}$. By what we previously showed it is imposible to get a 1 in the third position which is requred to make a move resulting in a 1 in the first position. \\
QED.\\
By symmetry $z_{p(b)}(b)+1\ge z_{p(b)}(f(b,x))$

\subsection{$w\circ f(b,x)\le w(b)+2$}
\begin{equation*}
\begin{split}
    w\circ f(b,x)&= z_{p(f(b,x))}(f(b,x))-z_{1}(f(b,x))\\
    &\le (z_{p(b)}(b)+1)-(z_{1}(b)-1)\quad \text{by 12}\\ 
    &= z_{p(b)}(b)-z_{1}(b)+2 \\
    &= w(b)+2\\
\end{split}
\end{equation*}

\subsection{$b = b'\cdot 0^3\cdot b'',\quad p\circ f(b,m_1+m_2) = p\circ f(b',m_1) + p\circ f(b'',m_2)$} 
Any valid move order in $b'$ and $b''$ is naturally also valid in $b$. 
Thus we can accieve equality by executing the same moves. 
We can never do better as, by 12, after m moves there can not be a 1 in the center as it is 2 away from any 1 in $b'$ and $b''$. This means that a 1 from $b'$ can never be adjacent to a 1 from $b''$ and thus no extra moves are available. 
\subsection{$b=0\cdot 1^n\cdot 0,\ p\circ o(b) = \lceil\frac{n}{2}\rceil$ }
Base case $n=1$: \\
There are no moves so $p\circ o(b)=1=\lceil\frac{1}{2}\rceil=\lceil\frac{n}{2}\rceil$\\
Base case $n=2$: \\
There is one move so $p\circ o(b)=1=\lceil\frac{2}{2}\rceil=\lceil\frac{n}{2}\rceil$\\
Inductive case $n\ge2$: \\
after one move we have $b = 0\cdot 1^{n-2}\cdot 001$ and since there is two zeroes between the group and the single one and the single one can not move closer to the group, they can not interact. So we can compute them seperately. By the inductive hypothesis we have that $p\circ o(0\cdot 1^{n-2}\cdot 0)=\lceil\frac{n-2}{2}\rceil$. There for we end up with $p\circ o(b)=p\circ o(0\cdot 1^{n-2}\cdot 0)+p\circ o(01)=\lceil\frac{n-2}{2}\rceil+1=\lceil\frac{n}{2}\rceil$\\
\end{document}

